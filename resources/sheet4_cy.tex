\documentclass[12pt]{article}

\usepackage{fullpage}
\usepackage{parskip}
\usepackage{setspace}
\usepackage{mathtools}
\usepackage{enumerate}
\usepackage{multicol}
\usepackage{booktabs}
\usepackage[T1,hyphens]{url}
\usepackage{hyperref}
\usepackage{minitoc}
\usepackage{standalone}
\usepackage{tikz}
\usepackage{array}
\usepackage{longtable}
\usepackage{pdfpages}
\usepackage{array}
\usepackage{pgfplots}
\usepackage{pgfplotstable}
\usepackage{color, colortbl}
\usepackage{diagbox}
\usepackage{titlesec}
\usepackage[gen]{eurosym}
\usepackage{multirow}
\usepackage{tcolorbox}
\usepackage[shortlabels]{enumitem}
\usepackage{thmtools}
\usepackage{xfrac}
\usepackage{amsfonts}
\usepackage{arydshln}
\usepackage{minted}
\usetikzlibrary{arrows}
\usetikzlibrary{decorations.markings}
\usetikzlibrary{shapes}
\usetikzlibrary{tikzmark}
\usepackage{fancyhdr}
\usepackage[
  headsep=1cm,
  headheight=2cm,
  footskip=1cm,
  bottom=2cm,
  top=2.5cm,
  lmargin=1cm,
  rmargin=2cm,
  marginparwidth=0cm
]{geometry}


\renewcommand\familydefault{\sfdefault}

\newcommand{\specialcell}[2][c]{%
  \begin{tabular}[#1]{@{}c@{}}#2\end{tabular}}

\newcommand{\specialcelll}[2][l]{%
  \begin{tabular}[#1]{@{}l@{}}#2\end{tabular}}

\newcommand{\specialcellr}[2][r]{%
  \begin{tabular}[#1]{@{}c@{}}#2\end{tabular}}

\newcommand*{\circled}[2][black]{
  \tikz[baseline=(char.base)]{
              \node[shape=ellipse,inner sep=2pt,
                draw=#1,
             ] (char) {#2};}
}


\newcommand*{\transportcell}[2]{%
\begin{center}
\resizebox{1.15\linewidth}{!}{%
\begin{tikzpicture}
  \draw[draw=none, fill=none] (0, 0) -- (0, 0.75) -- (0.75, 0.75) -- (0.75, 0) -- cycle;
  \node[align=right] at (0.375, 0.375) {\footnotesize{#1}};
  \node[align=right] at (0.7, 0.05) {\tiny{#2}};
\end{tikzpicture}%
}
\end{center}%
}

\declaretheorem[name=Datrysiad, shaded={bgcolor=gray!10, margin=10pt}]{sol}

\definecolor{codebg}{RGB}{255, 255, 230}

\nomtcrule

\begin{document}
\pagestyle{fancy}
\fancyhead{}
\fancyfoot{}
\fancyfoot[R]{\thepage}
\fancyhead[R]{MA2651 - Ymchwil Weithrediadol}
\fancyhead[L]{Taflen Problemau 4}


\begin{center}
\LARGE{\textbf{\textit{Taflen Problemau 4}}}
\end{center}

\vspace{10mm}

\begin{enumerate}
  \item Defnyddiwch efelychiad Monte Carlo i amcangyfrif gwerth yr integryn canlynol, gan rhoi cyfwng hyder ar gyfer eich amcangyfrif.
  \begin{equation*}
  I = \displaystyle{\int_{\sfrac{3}{2}}^{\sfrac{5}{2}} x^3 + 1 \; dx}
  \end{equation*}

  \vspace{10mm}

  \item Defnyddiwch efelychiad Monte Carlo i amcangyfrif gwerth yr integryn canlynol, gan rhoi cyfwng hyder ar gyfer eich amcangyfrif.

  \begin{equation*}
  I = \displaystyle{\int_{-1}^1 \sqrt{1 - x^2} \; dx}
  \end{equation*}

  \item Defnyddiwch efelychiad Monte Carlo er mwyn canfod y tebygolrwydd $p$, pan taflir pedwar dis ar yr un pryd, y gallant cael eu holltu i mewn i dwy bâr gyda symiau hafal. Er enghraifft, mae'r tafliad $(3, 1, 3, 5)$ yn gallu cael ei holltu i mewn i dwy bâr gyda symiau hafal: $3 + 3 = 1 + 5$, ond ni all $(2, 2, 2, 4)$. Rhowch cyfwng hyder ar gyfer eich amcangyfrif.

  \vspace{10mm}

  \item Mae'r system ar gyfer archebu bwyd mewn caffi yng nghanol Caerdydd yn ymddwyn fel rhwydwaith o giwiau. Mae tri cownter: y cownter bwyd oed, y cownter bwyd poeth, a'r til lle mae'r cwsmeriaid yn talu am eu bwyd. Os yw'r cwsmer eisiau bwyd oer yn unig, mae'n nhw'n ymuno a'r ciw bwyd oer; os ydynt eisiau bwyd poeth yn unig, mae'n nhw'n ymuno a'r ciw bwyd poeth; os ydynt eisiau bwyd oer a bwyd poeth, mae angen iddyn nhw ymuno a'r cownter bwyd oer cyntaf, ac yna mynd i'r cownter bwyd poeth. Ar ôl pigo lan eu bwyd, mae angen iddyn nhw hefyd ciwio wrth y til i dalu.
  \begin{itemize}
    \item Mae cwsmeriaid yn cyrraedd i'r cownter bwyd oer ar gyfradd 19 yr awr,
    \item Mae cwsmeriaid yn cyrraedd i'r cownter bwyd poeth ar gyfradd 12 yr awr,
    \item Mae 30\% o gwsmeriaid sy'n ciwio at y cownter bwyd oed hefyd eisiau bwyd poeth,
    \item Ar gyfartaledd mae'n cymryd 1 munud i weini bwyd oer, 2 a hanner munud i weini bwyd poeth, a 2 munud i dalu,
    \item Mae 1 gweinydd wrth y cownter bwyd oer, 2 weinydd wrth y cownter bwyd poerth, a 2 weinydd wrth y til,
    \item Mae'r caffi ar agor am 3 awr yn ystod yr amser cinio.
  \end{itemize}

  Mae'r caffi eisiau gwybod faint o gwsmeriaid byddant yn disgwyl gweini pob amser cinio ar gyfartaledd. Perfformiwch efelychiad digwyddiad-arwahanol gyda Ciw, a dangoswch y model cysyniadol.

  \vspace{10mm}

  \item Mewn adran achosion brys sydd ar agor am 24 awr, mae cleifion yn cyrraedd yn ôl proses Poisson ar gyfradd 35 yr awr. O rhain, caiff $\sfrac{1}{7}$ eu categoreiddio fel categori Brysbennu 1; $\sfrac{2}{7}$ fel categori Brysbennu 2; a $\sfrac{4}{7}$ fel categori Brysbennu 3. Mae cleifion yn aros i gael eu gweld gan un o'r 19 doctor sy'n gwetihio yn yr adran achosion brys, a chaiff eu gweld gan y doctor yn ôl ei flaenoriaethau: Mae gan cleifion Byrsbennu 1 blaenoriaeth dros cleifion Brysbennu 2, sydd yn ei tro yn cael blaenoriaeth dros cleifion Brysbennu 3. Mae'r amser mae'n cymryd i weld pob claf yn dibynnu ar ei categori brysbennu: caif cleifion Brysbennu 3 apwyntiad 15 munud; mae'r amser mae cleifion Brysbennu 2 yn gwario gyda doctor wedi'i dosrannu'n Unffurf rhwng 15 a 25 munud; ac mae'r amser mae cleifion Brysbennu 1 yn gwario gyda doctor wedi'i dosrannu'n Unffurf rhwng 20 a 90 munud. Beth yw'r amser aros cymedrig ar gyfer cleifion ym mhob categori brysbennu?

\end{enumerate}
\end{document}